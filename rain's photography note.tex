 \documentclass{book}

\usepackage{amsmath}
\usepackage{ctex}
\usepackage{color}
\begin{document}

\title{rain的摄影笔记}

\author{rain}

\maketitle

\tableofcontents%目录安排

\newpage
\section{二十九节摄影课从入门到精通}

\subsection{lesson1 认识曝光}

\textcolor{red}{进光量=镜头孔径*进光时间(快门速度)(感光度)(光圈大小)}

曝光合理:影响有质感、细节,明暗合理。

标准:亮有细节,暗有层次

通俗来说:亮处不能过曝,暗处不能漆黑一片
\subsection{lesson2 认识光圈}

光圈(f)——光圈数值越小,光圈越大(最佳光圈:该镜头最大光圈缩小2~3档)

作用:\begin{itemize}
  \item 调节进光量
  \item 调节景深效果
  \item 光圈影响画质
  \item 小光圈可以产生星芒(可以利用拍出特效)
\end{itemize}

对景深影响:

大光圈拍摄景深浅,主体清晰背景模糊;

小光圈拍摄景深长,主题和背景都很清晰。
\subsection{lesson3 认识快门}

\fbox{%
  \parbox{\textwidth}{%
    \begin{center}
      快速快门记录光的瞬间;
      慢速快门记录光的轨迹。
    \end{center}
  }%
}%给文字加边框
注:慢快门最好用上三脚架。
\subsection{lesson4 认识ISO}
ISO:感光度

iso 指相机硬件感光原件的能力。iso越高,接受光的能力越强,照片越亮。画质差,噪点多是因高ISO。有噪点的照片比拍虚了的照片有价值。

高ISO会使画面的颗粒感变得严重,带来更多的噪点(利用于星空摄影)
\subsection{lesson5 认识拍摄模式}

\begin{itemize}
  \item M档:手动曝光
  \item A(AV)档:光圈优先
  \item S(TV)档:快门优先
  \item P档:程序自动
  \item A(绿色框起来):全自动曝光模式
\end{itemize}
\subsection{lesson6 相机镜头的概念}

\begin{itemize}
  \item 定焦镜头:成像质量好,方便性差
  \item 变焦镜头:成像质量相对略差,但方便性好
\end{itemize}

\begin{itemize}
  \item 镜头越短,拍摄范围越广,视长角越大。
  \item 镜头越长,拍摄的范围越小,拍的距离越远。
\end{itemize}

\subsection{lesson7 镜头的焦距}

镜头的焦距

1,广角小于24mm。特点,取景范围大,多用于拍摄,风景,建筑等。

2,标准50mm左右。特点,和人眼看到的感觉相同,适合拍摄标准人像,静物,产品等。

3,中长焦,70-135mm左右,特别适合拍人像

4,长焦,大干135mm,特点取景范围小,适合拍动物,体育。焦距越长,背景越虚化。焦距越短背景越清楚,视长角越大。

\subsection{lesson8 对焦模式}

AF:自动对焦


尼康相机三种对焦方式的区别:\begin{itemize}
  \item AF-S:单次伺服自动对焦(半按快门释放键完成对焦后,若重新构图,对焦主体改变将进行重新对焦)(静止物体)
  \item AF-C:连续伺服自动对焦(半按快门释放键完成对焦后,若重新构图,对焦主体改变将进行重新对焦)(运动物体)
  \item AF-A:自动伺服自动对焦
\end{itemize}

MF:手动对焦

\subsection{lesson9 对焦点的作用}

单点区域(在选择的区域内对拍摄对象进行对焦,用于拍摄静止的拍摄对象

动态区域(手动选择对焦区域如果拍摄对象暂时离开所选对焦区域相机像根据来自其他区域的信息来进行对焦用于不规则运动的拍摄对象

不知道对焦对那里可选动态区域对焦,单点区域对于个别拍摄较多,成功率高。

\subsection{lesson10 相机白平衡}

由色温(K)决定,相反系数

色温高偏黄;色温低偏蓝。

AWB:自动白平衡,像人眼

\subsection{lesson11 运用白平衡}

色调分冷(青蓝紫)暖(红橙黄)。

标准是色温5200k,小于5200冷色调,大于5200暖色调。

应用:调大色温值,早上可以拍出傍晚的感觉

\subsection{lesson12 认识光质}

摄影是用光绘图。

光质(光的性质):\begin{itemize}
           \item 直射光(硬光)明暗分界明显,光的方向性强,加强物体立体感,通透感强,拍皱纹
           \item 散射光(柔光)几乎没有明暗反差,糖水片,画面柔和柔美朦胧,拍大雾天
         \end{itemize}

\subsection{lesson13 认识光位1}

光线位置使物体有明暗变化。

光位指光源相对于相机与被摄体的位置,既光线的方向与角度。

光位有五种:\begin{itemize}
        \item 顺光,光线来自被摄体正面,缺少立体感,证件照
        \item 侧光,位于被摄体侧面,立体感强,画面影调浓重、气氛强烈
      \end{itemize}

\subsection{lesson14 认识光位2}

光位有五种:\begin{itemize}
        \item 逆光,光源位于被摄体后面,逆光塑造物体轮廓,近黑远亮有神秘感,有艺术感。
        \item 顶光,光源位于被摄体上方,产生浓重向下阴影,显得沉重。
        \item 底光,光源来自被摄体下方,显得阴森恐怖。
      \end{itemize}

\subsection{lesson15 认识光比}

图像上明暗对比。整体明暗变化大,光比大。

逆光下,为环境中人物补光,调小光比,使人不至于变成剪影。

\subsection{lesson16 认识景别}

照片有主体/喜剧点。

\textcolor{blue}{远取其势,近取其神。}%蓝色字体

景别越小的时候,更重要的是取决于他的神态。

远景

全景(特写,近景,中景)

\subsection{lesson17 认识视角}

不同水平线的拍摄。
\begin{itemize}
  \item 高视角,如高山上往下拍,表现宽广遥远。适合拍摄宏大的场面,活动大场面,表现宽广、遥远。
  \item 低视角,仰拍,拍建筑,表现主体的高大。
  \item 平视角,平稳,稳重。
\end{itemize}

多观察,多练习。

\subsection{lesson18 认识透视}

运用画面的纵深感(道路)和距离感营造出三维效果。

透视:用线条/影调(明暗关系)在平面上表现物体空间的方法。

线性透视:近大远小,路的近宽远窄,汽车的近大远小,房子的近高远低,向心靠拢直致消失。

空气/大气透视:景物颜色的变化,近深远浅。

\subsection{lesson19 前景和背景的作用}

前景:焦点前方的景物。\textcolor{blue}{加强气氛},补充画面内容。增加画面层次感。\textcolor{blue}{均衡画面}。

背景:背景选择要合适。简洁干净的背景突出主题。背景丰富主题内涵。虚化简化背景突出主题。

\subsection{lesson20 平面布局和构图}

平面布局:把人、景、物安排在画面中以获得最佳布局,处理好画面中的点线面。

\begin{itemize}
  \item 黄金分割,主体放九宫格的相交点。
  \item 拍水平线,水平线要拍平。
  \item 三分法,平面上下两部分,要分主次,精彩的部分占画面的三分之二。
\end{itemize}

\textcolor{red}{如何学好构图:\begin{itemize}
         \item 多看好图片
         \item 构图从观察和模仿开始
         \item 把经典构图规则牢记于心
         \item 多练构图
         \item 打破规则寻求突破
       \end{itemize}}

\subsection{lesson21 常见的摄影题材和拍摄技法}

\textcolor{blue}{日出日落拍摄技巧}:\begin{itemize}
           \item 选择拍摄地点

           选择地点应该比较开阔,地势以高为宜。在野外拍摄的时候选择地势比较高的地方,比如高山上,拍摄的视角选择高处向下的角度,这样在取景时近处和地面就不会有什么多余的物体遮挡太阳,有利于主要内容的表现。重峦叠嶂在旭日东升时处于大逆光会产生一种层次丰富的效果。如果在城市中拍摄,可以选择一个比较高的楼,具有同样效果,层次也比较丰富。
           \item 拍摄时间

           日出日落的时间性很强,不同季节、不同时间段的表现都不一样,这就要求我们在拍摄之间必须很重视拍摄时间的选择。从季节方面来看,拍摄日出和日落最佳季节是春、秋两季。这两季比夏天的日出晚,日落早,对拍摄有利,在春秋云层较多,可增加拍摄的效果。
           \item 如何测光

           要较好地反映天空云霞的特点,应按云霞的中等亮度部位进行测光,还要注意回避太阳的强光影响到测光的准确。通常可选择“中央重点测光模式”对天空云霞的中等亮度部位进行测光,如果选择平均测光模式,很可能因较暗的地面景物而影响测光的准确性,造成曝光过度
           \item 注意构图

           在拍摄取景构图时,应该将太阳放在画面的趣味点上,并注意前景的选择和处理。在处理前景时,可选择有代表性的物体,如小树、小草、树枝、教室等等,这些前景在近光的照明下,常常以剪影的效果呈现于画面前,从而增强了画面的纵深效果。
         \end{itemize}

\textcolor{blue}{如何拍摄瀑布}:\begin{itemize}
                           \item 使用高速快门凝结住飞溅的水花和奔流的场景
                           \item 使用慢速快门拍摄雾状的水流和整体的水泻效果
                         \end{itemize}

一、合适天气:采用正确的天气和恰当的时间作为您的瀑布摄影,并且您能给予自己更多选择用更长的快门速度拍摄。

二、滤镜:使用减少光线直接进入相机,通常使用减光镜

三、低ISO:选择更低的ISO意味着您的相机传感器是敏感的和需要快门开放时间更长;它将意味将给您的拍摄更好的细节,更加清晰,减少颗粒。

四、构图要求:动静结合

\textcolor{blue}{烟花拍摄技巧}:\begin{itemize}
                           \item 如何记录烟花绽放的过程

                           设置小光圈,这样画面的景深范围大,清晰范围大;

                           使用手动对焦,对无穷远的地方对焦;

                           注意按下快门的时机,采用B门(手动)拍摄;

                           使用三脚架,保证稳定,完美的记录烟花轨迹。
                           \item 拍摄位置的选择

                           拍摄烟花应选择顺风位置,风可以将烟花产生的烟吹走,使画面保持干净;

                           可以运用湖面的倒影丰富画面层次,也可以利用城市的夜景作为衬托,避免画面单调
                         \end{itemize}

\subsection{lesson22 夜间人像拍摄}

1.内置闪光灯(相机自带、小小的)与外置闪光灯(独立的)的区别

夜景人像,逆光时,闪光灯补光

\begin{itemize}
  \item 内置闪光灯,有效距离5m,顺光无层次
  \item 外置闪光灯,有效距离10m,可模拟自然光,立体感层次感强
\end{itemize}

可以选择灯光效果的背景

一、 控制好环境光线

二 、闪光灯打亮主体

\subsection{lesson23 夜景风光拍摄及附件}

1)三脚架:拍晨昏、夜景、溪流、瀑布、烟火等必备的器材,有些要求严格的摄影同好拍风景一样要用,以确保影响质量;脚架的主要功能是防止相机拍摄时的震动,因此在选购是上脚架的稳固性就相对重要。

2)快门线:像遥控器一样。夜景拍摄需要长时间的曝光,比如烟花等等。拍摄时,减少相机的抖动,提高照片质量。

3)手电筒:照路。如果要拍摄银河、星轨等题材,就需要手电筒提供照明

4)遮光板:常常面临玻璃反光的问题。  做法是准备一个遮光板套在镜头上拍摄,用来消除玻璃反光光斑拍摄方法:大光圈,灯光,虚焦

\subsection{lesson24 常见滤镜使用效果}

\begin{itemize}
  \item 保护镜:适用: 所有题材  作用:防污、刮花,同时起到一定防潮作用
  \item UV镜:适用:所有题材  作用:可有效过滤掉紫外线提高照片的清晰度,提供更真实、艳丽的画面,同时还能起到一定保护作用
  \item 偏振镜/PL镜:适用:所有题材 作用:消除镜面反光,压暗天空、表现蓝天白云,同时降低快门速度;市场有线偏、圆偏两种规格,前者(指线偏)已经很少用到
  \item ND中灰密度镜:适用:风光摄影  作用:降低快门速度,防止过曝,适合在光照强烈的白天拍摄长曝光作品
  \item 硬边GND中灰渐变镜:适用:风光摄影 作用:压低天空亮度,平衡天空与地面、海洋的曝光;一般来说,有比较明显的分割线,适合拍摄地平线明显的题材。
  \item 软边GND中灰渐变镜:适用:风光摄影  作用:软边中灰渐变镜比较好控制,明暗渐变线并不明显,适合拍摄山体那种过度不是很平均、锐利的状况。
  \item RGND反向中灰渐变镜:适用:风光摄影  作用:区别RND中灰渐变镜,反向渐变滤镜是越靠中间越黑,有效阻挡画面中央的高光物体,平衡上下与中间部分的曝光。
  \item 彩色滤镜:适用:所有题材 作用: 更改画面颜色,不过由于数码照片强大的后期空间,这种滤镜已经不多见了。
  \item 冷调、暖调滤镜:适用:所有题材  作用:矫正色彩表现,或者通过更改白平衡达到改变照片氛围的作用。
  \item 近摄镜:适用:微距摄影  作用:缩短镜头对焦距离,放大被拍摄物体。
  \item 黑白滤镜:适用:所以题材  作用:黑白滤镜一般分为黄、绿、橙、红四种颜色,可以吸收不同波长的光线,从而呈现出不同效果的黑白照片。
\end{itemize}

\subsection{lesson25 顺光和测光使用方法}

顺光拍摄:顺光使被摄体均匀受光,整个景物的照度一致,没有明显的亮部和暗部,因而缺乏立体感,拍出的照片显得平淡、呆板,如果在强烈的阳光下拍摄,被摄者往往双眼眯成一条细缝

测光拍摄:侧光突出了景物上的明暗对比,受光部明亮,背光部分隐没在阴影中。侧光照射的景物显得层次丰富,立体感强,是风光摄影和人像摄影中常用的光线,但拍摄人像时要注意光比的变化。

\subsection{lesson26 逆光和漫射光使用方法}

逆光拍摄:被摄物体正面比较暗,边缘有一道明亮的轮廓线。如果被摄体处于一个暗背景,这道亮线就会把被摄体与前景截然分开,产生强烈的立体感。逆光拍摄要注意补光(反光板;外置闪光灯;内置闪光灯),注意光圈的大小,脸部和发丝不能过曝,\textcolor{blue}{剪影效果}。

漫射光:漫射光是吧晴天背阴处的反射光或阴天透过云层的光线,没有明显的方向,光线比较柔和,景物没有明显的亮部和暗部,反差柔和,适合拍集体照。

\subsection{lesson27 人像摄影构图}

\fbox{%
  \parbox{\textwidth}{%
    \begin{center}
      构图时主体的位置和所占比例大小
    \end{center}
  }%
}%给文字加边框

构图方法:\begin{itemize}
       \item 九宫格(三分法)
       \item 中心构图法:比较庄重、稳重
       \item 吊角构图
     \end{itemize}

表现形式:\begin{itemize}
       \item 虚实对比
       \item 大小对比
       \item 远近对比
       \item 色彩对比
       \item 剪影表现形式
     \end{itemize}
     
\subsection{lesson28 外景拍摄中不同拍摄角度的运用}

\begin{itemize}
  \item 平摄:平摄时镜头与被摄对象在同一水平线上,被摄人物没有明显的透视变形,视觉效果与人们正常观看事物的感受相同,拍摄效果显得自然、客观、平等、亲切。\textcolor{blue}{正常情况下拍摄人像,全身像时照相机高度等同于被摄对象的要腰部,半身像时照相机高度等同于被摄对象的胸部,特写时照相机高度等同于被摄对象的眼睛}。
  \item 俯摄:俯摄时镜头高于被摄对象的视平线,从高处向下拍摄,这时被摄对象显得低矮且身体压缩,如果是近景头像,会出现额头夸大而下巴窄小,视觉上可给人清秀之感。
  \item 仰摄:仰摄时镜头低于被摄对象的视平线,从低处向上拍摄,这时拍摄对象显得高大且身体修长,如果是近景全身,会出现脚长身短,视觉上可给人高大挺拔感觉。
\end{itemize}

\subsection{lesson29 摄影的影调}

\textcolor{blue}{影调:在不同强弱的光线照射下,被摄物体产生不同的明暗反差}

高调的拍摄方法:\begin{itemize}
          \item 用浅色或白色的背景,服装道具,亮的(白的)占80\%以上,暗的占20\%以下。
          \item 一般用正面光(顺光)拍摄,光比不大于1:1.5,采用浅灰色的轮廓线将人与背景分离。
          \item 用数码相机拍摄应准确曝光,可在后期适当提高照片的亮度。
          \item 高调作品一般以上半身特写居多,背景亮度亮于主体1.5倍(0.5-1.5档之间)
        \end{itemize}
        
低调的拍摄方法(景别以特写,半身为主):\begin{itemize}
          \item 背景,服装道具应选用黑颜色或深颜色为主。
          \item 用侧光或侧逆光布光,光比控制在1:3.5以上,暗的占80\%以上,亮的占20\%以下。
          \item 拍摄时可按亮部准确曝光或过0.5档曝光。
          \item 用轮廓光打亮头发,让主体与背景分离。
        \end{itemize}

\section{summary}

摄影是光与影的艺术



\section{vlog西瓜视频万粉训练营}

\subsection{lesson1}
封面,标题,关键词

一、如何获得收益:\begin{itemize}
  \item 实名认证
  \item 发布
  \item 申请原创权限
  \item 通过原创审核
  \item 我的$\rightarrow$创作中心$\rightarrow$收益概览
\end{itemize}

二、如何录制

16:9 横屏(可去剪映验证一下)

三、录制什么视频

vlog(生活记录)

四、特点:人格化、真实性、生活记录\begin{itemize}
    \item 名字要让人记得住,听得懂,且有自己的人设
    \item 本人出镜,自身说话讲述,无滤镜、美颜
    \item 自己生活中的日常记录
    \item 视频时长:1min++(3到5min最佳)
    \item 加字幕
  \end{itemize}

头条官网:mp.toutiao.com

\subsection{lesson2}
一、开头:\begin{itemize}
       \item hello,大家好,我是×××,我要去做×××
       \item hello,大家好,我是×××,我现在在×××,带大家看看×××
     \end{itemize}

视频中不允许出现:\begin{itemize}
           \item 视频模糊
           \item 视频尺寸错误
           \item 视频中有其他软件水印及边框
           \item 环境太黑,视频曝光
         \end{itemize}

二、真人说话+字幕\begin{itemize}
           \item 真人说话,即录制时候说话
           \item 也可后期配说话声和配音乐
         \end{itemize}

\textcolor{blue}{方法:打开剪映,点击开始创作$\rightarrow$选中视频添加到项目$\rightarrow$点击音频关闭原声$\rightarrow$点击录音$\rightarrow$按住录,进行配音点击√$\rightarrow$点击导出视频,配音就ok了}

添加字幕

\textcolor{blue}{打开剪映,点击开始创作$\rightarrow$选中视频添加到项目$\rightarrow$点击文本$\rightarrow$识别字幕$\rightarrow$开始识别}

三、标题\begin{itemize}
      \item 最佳字数28到30字
      \item 三段式最佳
    \end{itemize}
    1、突出实用性\begin{itemize}
             \item 物品实用性
             \item 干货内容
           \end{itemize}
    2、留悬念\begin{itemize}
           \item 引人进行猜想
           \item 吸引人讨论
         \end{itemize}
    $\star$3、数字量化细节\begin{itemize}
              \item 数字产生对比
              \item 突出数字
            \end{itemize}

四、封面\begin{itemize}
      \item 作用:视频预告+标题补充
      \item 画面有主色调,形成自身风格
      \item 封面压字,突出视频重点,2到6字左右
    \end{itemize}

\subsection{lesson3}

选题两步走

第一步:选题方法\begin{itemize}
          \item 提前注意节点(节假日,节气,生日,纪念日,赛事)
          \item 观察其他人发了啥
          \item 关注热点
        \end{itemize}
第二步:挑选优质选题

脚本的设计:介绍————内容————总结

手机拍摄思路和技巧\begin{itemize}
           \item 光线一定要充足,画面一定要防抖
           \item 画面构图技巧:人在画面$\frac{2}{3}$的位置最佳
           \item 拍摄技巧,尝试各难度拍摄(一边后退一边抬升,画面更加鲜活)
         \end{itemize}

视频快速剪辑(剪映)

\textcolor{blue}{导入$\rightarrow$删除段落$\rightarrow$识别字幕$\rightarrow$转场$\rightarrow$添加背景音乐$\rightarrow$导出}

\subsection{lesson4}

注意事项:\begin{itemize}
       \item 1、\begin{itemize}
                 \item 不出镜/出镜跟观众没有互动
                 \item 只对镜头说话,没有其他任何场景
                 \item 出镜说话,也有场景,但内容割裂
               \end{itemize}
       \item 2、收音和配乐音量不平衡(剪映声音跟随功能)
       \item 3、\begin{itemize}
                 \item 标题不够明确
                 \item 内容要有头有尾,讲+看结合
               \end{itemize}
       \item 4、视频素材问题\begin{itemize}
                       \item 长期发非生活记录的内容
                       \item 多次发同内容的视频
                       \item 镜头晃动太严重
                     \end{itemize}
     \end{itemize}

\subsection{lesson5}

心态调整:\begin{itemize}
       \item 生活记录是录制出来的,不是刻意拍出来的
       \item 用剪辑、标题、封面去丰富内容
       \item 降低前期拍摄心理压力,随手拍
       \item 现在的生活很美好,学会记录
     \end{itemize}

\begin{itemize}
  \item 贯活用好脚本————学会创造想法,作好记录
  \item 处理素材————学会拆分和整合
  \item 为内容和人设服务————强调不同亮点
\end{itemize}

\subsection{lesson6}

吸引更多粉丝

内容上:\begin{itemize}
      \item 知识干货分享
      \item 代替粉丝发声,塑造良好形象
      \item 为粉丝定制内容,视频片尾询问观众想要看什么内容
    \end{itemize}

环节设计:\begin{itemize}
       \item 结尾提示观众转发、留言、收藏、关注
       \item 走心的回复粉丝的评论
       \item 不定期给粉丝发小福利,抽奖活动,结尾公布上一期中奖的关注
     \end{itemize}

定期更新:\begin{itemize}
       \item 更新频率+每天固定时间
       \item 结尾对下期做预告
     \end{itemize}

\subsection{lesson7}

人设定义:人物设定或人物标签

人设效果:看到你的画面、声音、内容、名称就会想到这个人就是你

制定人设原因:打造差异化形成独立标签性/记忆点,提升辨识度,吸引流量

人设外展:\textcolor{blue}{昵称/简介/标题/封面/内容},也包括形象、穿着、头发、眼神、手势、行为等

如何形成人设:\begin{itemize}
         \item 发掘人设定位
         \item 制定开场语
       \end{itemize}

发崛自己的特点和爱好擅长,选择自己的人设定位。

特点:\begin{itemize}
     \item 外貌:酒窝、虎牙、长腿
     \item 性格:爱笑、开朗、勤劳、乐观、冒险、好奇、认真
   \end{itemize}

爱好擅长:爱运动、旅行、读书、跳舞、画画、唱歌、跳舞、滑板

人物标签:留学生、考研党、医学生、体育生、运动达人、博士

制定开场语,原因:自我介绍,加深别人对你的印象,突出自己的人设。

\subsection{lesson8}

如何让自己的视频提升内容价值:\begin{itemize}
                 \item 视频具备稀缺性:具有稀缺性,形成绝对优质/相对竞争优质
                 \item 视频具有传播特点:\begin{itemize}
                                  \item 价值观正确
                                  \item 简易上口易传播
                                  \item 洞悉用户需求,戳到共鸣点
                                \end{itemize}
                 \item 视频具备核心价值:视频可满足用户的需求$\rightarrow$它具备核心价值(1知识供给2情感3娱乐)
               \end{itemize}

脚本策划的小亮点:\begin{itemize}
           \item 技巧一:有内容价值
           \item 技巧二:增添亮点;标题和内容有对比
         \end{itemize}

一、封面风格设定

\begin{table}[h]
\caption{封面风格设定}
  \centering
\renewcommand\arraystretch{2}
\begin{tabular}{|c|c|c|}
  \hline
  % after \\: \hline or \cline{col1-col2} \cline{col3-col4} ...
  内容方向&封面建议色系&封面场景/突出感受\\
  \hline
  学习类/严谨学术派&简洁清晰(蓝白色系等)&图书馆、会议室、实验室、西装、校服\\
  \hline
  勤学奋斗/乐观积极&青绿黄等&色系突出积极乐观,青春洋溢\\
  \hline
  生活享受&粉蓝红色&餐厅,展厅,商场,机场,海边;风景,酒店\\
  \hline
  欢脱搞笑&黄红蓝等&突出;搞怪,惊讶,特别,新奇\\
  \hline
\end{tabular}
\end{table}

二、画面风格设定\begin{itemize}
          \item 环境:布局,灯光,配色,构图
          \item 视频质感:朦胧,清新,质感,高级
        \end{itemize}

三、语言风格设定\begin{itemize}
          \item 解说风格
          \item 配乐
        \end{itemize}

\subsection{lesson9}

\begin{table}[h]
\caption{拍摄技巧}
  \centering
\setlength{\tabcolsep}{7mm}{
\begin{tabular}{|c|}
  \hline
  % after \\: \hline or \cline{col1-col2} \cline{col3-col4} ...
  推:向前推镜头\\
  \hline
  拉:向后拉镜头\\
  \hline
  摇拍:移动拍摄\\
  \hline
  俯:向下拍摄\\
  \hline
  仰:向上拍摄\\
  \hline
  升:侧面向上移动\\
  \hline
  降:侧面向下移动\\
  \hline
\end{tabular}}
\end{table}

剪辑技巧(剪映)\begin{itemize}
          \item 多轨道剪辑(文字、声音、贴纸、特效、视频)
          \item 视频动画
          \item 定格(突出)
          \item 转场
          \item 音效
          \item 卡点(需使用剪映自带视频,音频)
          \item 文本贴纸
          \item 画中画
          \item 滤镜
        \end{itemize}

\end{document}



